\documentclass[document.tex]{subfiles}
\begin{document}
\chapter{Experimental Evaluation}
\section{Introduction}
The experiments are complicated by the fact that \textbf{PWPS} is limited to single equations of percentage word problems, and ALGES
can only handle single-equations algebra problem with only addition, subtraction, multiplication and division. Our main experimental result is to find the solution of Percentage word problems.

For experiment our system \textbf{PWPS}, there is two parts. One is Experimental Setup and the other is Dataset. In the following section, I have discussed those in details.
\section{Experimental Setup}
We use the Stanford De-
pendency Parser in CoreNLP 3.4\cite{32} to obtain syntactic information used for
grounding and feature computation. For the ILP
model, we use CPLEX 12.6.1 (IBM ILOG, 2014)\cite{33}
to generate the top $M = 100$ equation trees with
a maximum stack depth of 10, aborting exploration
upon hitting $10K$ feasible solutions or $30$ seconds.
We use Python’s SymPy package for solving equations for the unknown. For the local and global models,we use Random forest classifier \cite{34,35}(Discussed in Appendix A).
We have used \textbf{Linux (Ubuntu 16.04 LTS)} as our operating system. So, the following process shown only for Linux Environment.
%\subsection{Java Installation}
%For installing Java Compiler, the following command should run from the terminal.
%\begin{center}
%	\begin{lstlisting}
%	sudo apt-get install default-jdk
%	sudo apt-get install default-jre
%	\end{lstlisting}
%\end{center}
%
%\subsection{Python v2.0 and PIP}
%We used \textbf{python 2} as our programming language. For installing Python 2 and Pip, the following command should run from the terminal.
%\begin{center}
%	\begin{lstlisting}
%	sudo apt-get install python2
%	sudo apt-get install pip2
%	\end{lstlisting}
%\end{center}
%
%\subsection{Stanford Dependency Parser CoreNLP 3.4 Server}
%For parser, we have used Stanford Dependency Parser CoreNLP 3.4. This is available at \url{http://nlp.stanford.edu/software/stanford-corenlp-full-2014-06-16.zip}
%
%\subsection{Running Server}
%Following command should run for installing the server essentials and other packages
%\begin{lstlisting}
%sudo pip install pexpect unidecode jsonrpclib 
%git clone https://bitbucket.org/torotoki/corenlp-python.git
%cd corenlp-python
%wget http://nlp.stanford.edu/software/stanford-corenlp-full-2014-08-27.zip
%unzip stanford-corenlp-full-2014-08-27.zip
%\end{lstlisting}
%Then, to launch a server:
%\begin{lstlisting}
%python corenlp/corenlp.py
%\end{lstlisting}
%Optionally, you can specify a host or port:
%\begin{lstlisting}
%python corenlp/corenlp.py -H 0.0.0.0 -p 3456
%\end{lstlisting}
%That will run a public JSON-RPC server on port 3456.
%And you can specify Stanford CoreNLP directory:
%\begin{lstlisting}
%python corenlp/corenlp.py -S stanford-corenlp-full-2014-08-27/
%\end{lstlisting}
%\subsection{PWPS Running}
%To get our code run the following command:
%\begin{lstlisting}
%git clone https://github.com/habibrahmanbd/PWPS
%\end{lstlisting}
%After that, for running our system:
%\begin{lstlisting}
%cd PWPS/
%./PWPS problemset.json
%\end{lstlisting}
%where, \textbf{$problemset.json$} is the dataset.

\section{Dataset Description}
This work deals with percentage word problems that map to single equations with varying length. Every equation may involve multiple math operations including multiplication, division, subtraction, and addition over non-negative rational numbers and one variable. A sample data in JSON format is shown in the Fig. \ref{fig:data} .
\begin{figure}[H]
	\begin{center}
		\begin{lstlisting}
{
	"iIndex": 121,
	"lSolutions": [ 28.0 ],
	"lEquations": [ "x=56.0*(50.0/100.0)" ],
	"sQuestion": " Beth took a math quiz last week. There were 56 problems on the quiz and Beth answered 50% of them correctly. How many problems did Beth get correct? "
}
		\end{lstlisting}
	\end{center}
	\caption{Sample Dataset in JSON Format}
	\label{fig:data}
\end{figure}

We collected a new dataset from \url{http://math-aids.com}, \url{http://ixl.com}, \url{https://www.khanacademy.org} and \url{http://algebra.com}. Dataset statistics is given below in TABLE \ref{tab:datasetstat}.

\begin{table}[H]
	\caption{Dataset Statistics}
	\begin{center}
		\begin{tabular}{|l | r|}
			\hline
			\textbf{Statistics}& \# \\ \hline
			Number of Problems in Dataset& 185 \\
			Number of Sentences in Dataset& 592\\
			Number of Words in Dataset& 5698\\
			Average Sentences per Problem& 3.2\\
			Average Words per Problem& 30.8\\
			\hline
		\end{tabular}
	\end{center}
	\label{tab:datasetstat}
\end{table}

\section{Implementation Procedure}
\end{document}
